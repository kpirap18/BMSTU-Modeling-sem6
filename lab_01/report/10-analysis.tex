\chapter{Теоретические сведения}

\textbf{Тема}

Программная реализация приближенного аналитического метода и численных алгоритмов первого и второго порядков точности при решении задачи Коши для ОДУ.

\textbf{Цель работы}

Получение навыков решения задачи Коши для ОДУ методами Пикара и явными методами первого порядка точности (Эйлера) и второго порядка точности (Рунге-Кутта).


\textbf{Исходные данные}

ОДУ, у которого отсутствует аналитическое решение:

\begin{equation}
    \label{initial_odu}
    \begin{cases}
        u'(x) = x^2 + u^2\\
        u(0) = 0
    \end{cases}
\end{equation}\newline

На вход подается конечное значение \textit{x\_max} и шаг (в реализованном программе данные задаются константами).

\textbf{Результаты}
\begin{enumerate}
	\item Таблица, содержащая значения аргумента с заданным шагом в интервале \textit{[0, x\_max]} и результаты расчета функции	u(x) в приближениях Пикара (от 1-го до 4-го), а также численными методами. Границу интервала x\_max выбирать максимально возможной из условия, чтобы численные методы обеспечивали точность вычисления решения уравнения u(x) до второго знака после запятой. 
	\item График функции в диапазоне \textit{[-x\_max, x\_max]}.
\end{enumerate}

\section{Решение}

\textbf{Описание} 

Обыкновенными дифференциальными уравнениями (ОДУ) называются
уравнения с одной независимой переменной. Если независимых переменных больше, чем одна, то уравнение называется дифференциальным уравнением с частными производными.

\textbf{Задача Коши}

Общее решения ДУ n-го порядка зависит от констант общего решения. Для выделения частного решения требуется задать n условий.

В задаче Коши все дополнительные условия задаются в одной точке:

\begin{equation}
{\begin{cases}
	u'(x) = f(x,u) \\
	u(\xi) = \eta
	\end{cases}}
\label{eq:ref1}
\end{equation}


Можно выделить три метода решения обыкновенных дифференциальных уравнений в задаче Коши: аналитические, аналитические приближенные и численные.


Для решения данного ОДУ были использованы 3 алгоритма.

\subsection{Метод Пикара}

Имеем:

\begin{equation}
    \label{solution}
    u(x) = \eta +  \int_{\xi}^{x} f(t,u(t)) \,dt
\end{equation}

Строим ряд функций:

\begin{equation}
    \label{sol}
    y^{(s)} = \eta +  \int_{\xi}^{x} f(t,y^{(s-1)}(t)) \,dt, \quad \quad
    y^{(0)} = \eta
\end{equation}

Построим 4 приближения для уравнения (\ref{solution}):

\begin{equation}
    \label{f1}
    y^{(1)}(x) = 0 + \int_{0}^{x} t^2 \,dt = \frac{x^3}{3}
\end{equation}

\begin{equation}
    \label{f2}
    y^{(2)}(x) = 0 + \int_{0}^{x} (t^2 + \left(\frac{t^3}{3}\right)^2) \,dt = \frac{x^3}{3} + \frac{x^7}{63}
\end{equation}

\begin{equation}
    \label{f3}
    y^{(3)}(x) = 0 + \int_{0}^{x} (t^2 + \left(\frac{t^3}{3} + \frac{t^7}{63}\right)^2) \,dt = \frac{x^3}{3} + \frac{x^7}{63} + \frac{2x^{11}}{2079} + \frac{x^{15}}{59535}
\end{equation}

\begin{equation}
    \begin{split}
        \label{f4}
        y^{(4)}(x) = 0 + \int_{0}^{x} (t^2 + \left(\frac{t^3}{3} + \frac{t^7}{63} + \frac{2t^{11}}{2079} + \frac{t^{15}}{59535}\right)^2) \,dt = \frac{x^3}{3} + \frac{x^7}{63} + \frac{2x^{11}}{2079} +\\
        \frac{x^{15}}{59535} + \frac{2x^{15}}{93555} + \frac{2x^{19}}{3393495} + \frac{2x^{19}}{2488563} + \frac{2x^{23}}{86266215} + \\
        \frac{x^{23}}{99411543} + \frac{2x^{27}}{3341878155}  + \frac{x^{31}}{109876902975}
    \end{split}
\end{equation}

В программе используется также 5 приближение. Его подсчеты приведены на листе бумаге, приложенному к отчету.
 
\subsection{Метод Эйлера}

\begin{equation}
    \label{ey}
    y^{(n+1)}(x) = y^{(n)}(x) + h \cdot f(x_{n}, y^{(n)})
\end{equation}

\indent Порядок точности: $O(h)$.

\subsection{Метод Рунге-Кутта}

\begin{equation}
    \label{rk}
    y^{n+1}(x) = y^{n}(x) + h ((1-\alpha) R_1 + \alpha R_2)
\end{equation}\newline

где $R1 = f(x_{n}, y^{n})$, $R2 = f(x_{n} + \frac{h}{2\alpha}, y^{n} + \frac{h}{2\alpha}R_1)$, $\alpha = \frac{1}{2}$ или 1\newline

Порядок точности: $O(h^2)$.

\section{Результаты}

Результаты представлены для начального значения \textit{x = 0}, конечное значение \textit{x\_max = 2}, шаг \textit{step = 1e-4}.

Выводится часть результатов с шагом 0.05.

Позиция столбцов в таблице. 

\begin{enumerate}
	\item Значение Х.
	\item Метода Пикара:
	\begin{enumerate}
		\item Первое приближение;
		\item Второе приближение;
		\item третье приближение;
		\item четверное приближение.
	\end{enumerate}
	\item Метод Эйлера;
	\item Метод Рунге-Кутты.
\end{enumerate}
\img{150mm}{res}{Результаты}

