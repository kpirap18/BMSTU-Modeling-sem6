\section{Ответы на вопросы}


\textbf{Вопрос 1} Укажите интервалы значений аргумента, в которых можно считать решением заданного уравнения каждое из первых 4-х приближений Пикара, т.е. для КАЖДОГО приближения указать свои границы применимости. Точность результата оценивать до второй цифры после запятой. Объяснить свой ответ.
	
Первое приближение Пикара можно считать решением уравнения до тех пор, пока совпадают результаты для первого и второго приближений до второго знака после запятой. Второе приближение Пикара можно считать решением уравнения до тех пор, пока совпадают результаты для второго и третьего приближений до второго знака после запятой. Третье приближение Пикара можно считать решением уравнения до тех пор, пока совпадают результаты для третьего и четвертого приближений до второго знака после запятой. Четвертое приближение Пикара можно считать решением уравнения до тех пор, пока совпадают результаты для четвертого и пятого приближений до второго знака после запятой.

На рисунке \ref{img:p1} представлены результаты для первого приближения.

\img{80mm}{p1}{Первое приближение}
\newpage

На рисунке \ref{img:p2} представлены результаты для второго приближения.

\img{100mm}{p2}{Второе приближение}
\newpage

На рисунке \ref{img:p3} представлены результаты для третьего приближения.

\img{120mm}{p3}{Третье приближение}
\newpage

На рисунке \ref{img:p4} представлены результаты для четвертого приближения.

\img{100mm}{p4}{Четвертое приближение}
\newpage


\textbf{Вопрос 2} Пояснить, каким образом можно доказать правильность полученного результата при фиксированном значении аргумента в численных методах. 

Доказать правильность полученного результата при фиксированном значении аргумента в численных методах можно посредством постепенного уменьшения шага. Если при уменьшении шага полученный результат изменится незначительно (относительно предыдущих изменений) или не изменился совсем, то полученный результат можно считать правильным.

\textbf{Вопрос 3} Каково значение решения уравнения в точке x=2, т.е. привести значение u(2).

Ответ можно получить, опираясь на вопрос 2 (выше). Посредством уменьшения шага и анализом изменения полученного численным методом значения. 

Для $10^-2$:
\img{10mm}{2}{}

Для $10^-3$:
\img{10mm}{3}{}

Разница с предыдущем составляет $171.9864$


Для $10^-4$:
\img{10mm}{4}{}

Разница с предыдущем составляет  $16.8276$

Для $10^-5$:
\img{10mm}{5}{}

Разница с предыдущем составляет  $0.2299$

Для $10^-6$:
\img{10mm}{6}{}
\newpage

Разница с предыдущем составляет  $0.0024$

Таким образом: $u(2) = 317.72$

\textbf{Вопрос 4} Дайте оценку точки разрыва решения уравнения.

\textbf{Вопрос 5} Покажите, что метод Пикара сходится к точному аналитическому решению уравнения 

	\begin{equation}
	\label{initial_odu}
	\begin{cases}
	u'(x) = x^2 + u\\
	u(0) = 0
	\end{cases}
	\end{equation}\newline

